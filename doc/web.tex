\documentclass[12pt]{article}
\usepackage[utf8]{inputenc}
\usepackage[czech]{babel}
\usepackage{graphicx}

\begin{document}
\begin{titlepage}
\begin{center}
\mbox{} \\[3cm]
\huge{Semestrální práce z předmětu KIV/WEB} \\[2cm]
\Large{Tomáš Maršálek} \\
\large{marsalet@students.zcu.cz} \\[1cm]
\normalsize{\today}
\end{center}
\end{titlepage}

\section{Použité technologie}
\subsection{html}
Použití html je jedním z povinných požadavků práce. Použitá verze je HTML4,
některé prvky (pouze několik) jsou podporovány až verzí HTML5. Dynamicky
vygenerované stránky jsou HTML5 validní.

\subsection{css}
Vzhled webových stránek je celkem jednoduchý, ale přehledný. Nemá problém se
zmenšováním okna nebo při zoomování přímo v prohlížeči. Veškeré pozicování,
velikosti písma, barvy, atd. jsou zajištěny výlučně kaskádovými styly.
Stylování přímo v html je dnes zastaralé. Kvůli čitelnosti jsou všechny styly
ve zvláštním souboru, nikde není použit inline zápis typu 
\verb|<tag style="...">|. 

Obtížnými prvky bylo stylování formulářů a~spodního panelu. Naštěstí se
stylováním formulářů zabývá poměrně mnoho lidí, proto nebylo těžké vyhledat
správný způsob zarovnání pomocí samotného css bez použití tabulek. Tabulkové
rozložení stránek je dnes kontroverzní téma, kdy jedna skupina tvrdí, že
tabulky neslouží k uspořádání prvků na stránce, když už máme dostatečně silné
kaskádové styly, které stejnou věc zvládnou přehledněji, například pomocí
$float$ vlastnosti.  Výhodou $float$ oproti tabulkovému rozložení je i lepší
přizpůsobivost při manipulaci s velikostí okna nebo přiblížením.  Proto mé
rozhodnutí nepoužívat tabulky k tomuto účelu.

Další problém s neintuitivním řešením je spodní panel přilepený
k dolnímu okraji obrazovky (sticky footer). Opět se naštěstí našlo pár návodů,
jak tohoto docílit.


\subsection{php}
Klíčovou technologií celé práce je právě php. Velké množství kódu je ve
stránkách vygenerováno dynamicky, protože obsah je závislý na oprávnění
uživatele k~přístupu k~tomuto obsahu. Například běžný nezaregistrovaný uživatel
podle zadání nemá přístup k~seznamu členů klubu. Ředitel klubu má nejvyšší
oprávnění a~má tedy možnost vkládat nové aktuality, odehrané zápasy, přidávat
nebo upravovat hráče a má k tomu k dispozici formuláře na příslušné stránce.
Tyto prvky musí být zajištěny na straně serveru tak, že každý uživatel uvidí
nejvýše tolik, kolik jeho přístupová práva dovolují.

Důležité je zajistit bezpečnost a stabilitu webových stránek. Jsou ošetřeny
případy, kdy by se uživatel dostal omylem nebo se zlým úmyslem na některou
stránku jinak, než povoleným způsobem. Například kdyby chtěl upravit jiného
uživatele jednoduchým přepsáním GET parametru v url. I kdyby se uživateli
podařilo odeslat požadavek na změnu údajů, při vyhodnocování požadavku stejně
dojde ke zkontrolování povolení k přístupu, proto není možné, aby někdo
manipuloval s cizími daty. Samozřejmě za předpokladu, že vše funguje tak jak
má, což obecně není jednoduchá záležitost.

Kontrola údajů zadaných pomocí formuláře probíhá i na straně klienta,
ale hlavní kontrola je samozřejmě ta na straně serveru. Ta však dává upozornění
pouze v těch případech, kdy nebylo možné provést kontrolu u klienta. To je
například při vkládání uživatele, jestli již takový existuje.

\subsection{SQLite 3}
Databázový systéme je zde implementace SQL jazyka SQLite 3. Zvolil jsem ji pro
svoji jednoduchost oproti známémějšímu a rozšířenějšímu MySQL. Webové stránky
velikosti této semestrální práce by s přehledem obsloužil jednoduchý xml
soubor, ale zadání vyžaduje použití databáze, proto nejodlehčenější variantou
je právě SQLite. Tato implementace je opravdu jednoduchá na používání, což je
její největší výhodou. Výhodou je také netřeba databázového serveru, protože
celá databáze je po celou dobu uložena v jediném souboru. Přenositelnost
databáze je tedy zcela triviální.

\subsection{ostatní}
Kontrola formulářů je velmi efektivní při použití javascriptu a ještě
efektivnější v kombinaci s AJAXem. Rozhodl jsem se ale vyzkoušet nové
formulářové prvky HTML5, které poskytují jednoduchou kontrolu a ještě
jednodušší obsluhu formulářů. Například při vstupu data nebo čísla se opravdu
usnadní práce jak programátora, tak uživatele při používání onoho formuláře.
Podpora těchto prvků zdá se být dnes na přijatelné úrovni.

\section{Adresářová struktura a architektura}
Webové stránky mají poměrně jednoduchou architekturu. Použil jsem takové
postupy a technologie, aby komplexita architektury odpovídala komplexitě
stránek tak, jak jsou vnímány uživatelem. Zdrojové kódy jsou v kořenovém
adresáři a kaskádové styly v adresáři $css$. Protože nebyl použit javascript
ani jiné prvky zajišťující logiku na straně klienta, jiné adresáře nejsou ani
potřeba. Databáze je uložena v souboru $data.db$ rovněž v kořenovém adresáři.

\subsection{architektura}
Rozvržení stránek bylo zvoleno tak, aby co nejvíce šetřilo zdrojovým kódem
a bylo tedy co nejjednodušší při jakýchkoliv případných změnách. Uživatel je
vždy přítomen na stránce $index.php$, obsah je pouze dynamicky vygenerován
podle GET parametru $id$. Toto rozvržení umožňuje ponechat zdrojový kód horního
a dolního panelu stejný a pouze měnit obsah.

Každá stránka s obsahem má vždy dva účely (s vyjímkou stránky $kontakt.php$,
která je víceméně pouze statického charakteru. Prvním účelem je zobrazit obsah
tak, jak ho získáme z databáze. Druhý účel zajišťuje logiku celé stránky. Tedy
zajišťuje vyhodnocování formuláře té stejné stránky. V případě úspěchu při
vyhodnocování formuláře přesměruje uživatele na stejnou stránku, ale tak, aby
se pouze zobrazil obsah. Tento styl je poměrně hojně využívaný, protože
umožňuje ponechat logiku i obsah jedné stránky v jednom souboru.

\section{uživatelská příručka}
Stránky obsahují kategorie úvod, výsledky, kontakt a pro přihlášené uživatele
sekci členové. Na úvodní stránku se dostaneme kliknutím na nadpis \uv{Kopáči}.
K úpravě profilu se dostaneme kliknutím uživatelské jméno, v případě, že jsme
přihlášeni. Bude nám poskytnut formulář ke změně údajů a hesla. Pokud nechceme
měnit heslo, stačí ho nevyplňovat. Zbytek stránek je víceméně intuitivní.

\section{závěr}
Kódování stránek je UTF-8, čeština proto funguje bez nejmenších problémů.
Práce byla úspěšně otestována v prohlížečích Mozilla Firefox 8.0 a Chromium
15.0.  Při tvorbě webových stránek jsem se naučil používat jazyk php pro
dynamické generování obsahu na straně serveru a manipulaci s relačními
databázemi pomocí jazyka SQL. Svoji znalost jazyka markovacího html jsem
rozšířil o nové prvky z HTML5.

\end{document}
